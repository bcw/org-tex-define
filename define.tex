% Created 2016-08-25 Thu 19:27
\documentclass[11pt]{article}
\usepackage[utf8]{inputenc}
\usepackage[T1]{fontenc}
\usepackage{fixltx2e}
\usepackage{graphicx}
\usepackage{longtable}
\usepackage{float}
\usepackage{wrapfig}
\usepackage{rotating}
\usepackage[normalem]{ulem}
\usepackage{amsmath}
\usepackage{textcomp}
\usepackage{marvosym}
\usepackage{wasysym}
\usepackage{amssymb}
\usepackage{hyperref}
\tolerance=1000
\RequirePackage{fancyvrb}
\DefineVerbatimEnvironment{verbatim}{Verbatim}{fontsize=\scriptsize}
\newcommand{\myvqeqn}{y = 2(x-1)^2+1}
\newcommand{\mat}[1]{\mathbf{#1}}
\author{Brian C. Wells}
\date{\today}
\title{Org mode \LaTeX{} macros (for both HTML and \LaTeX{} export)}
\hypersetup{
  pdfkeywords={},
  pdfsubject={},
  pdfcreator={Emacs 24.5.1 (Org mode 8.2.10)}}
\begin{document}

\maketitle
\tableofcontents

This document should be exported to HTML and \LaTeX{} to check that the
proper code is generated.  A PDF file should also be available, but
looks a bit bad because Org mode macros \emph{must} be written on a single
line, and some of these macros overfill the line (even in a fairly
small font).

This Literate Program is an Org mode setup file that makes it easy to
define \LaTeX{} macros that work in \emph{both} \LaTeX{} and HTML export.  To
"tangle" the Org mode source code for this document to generate the
\verb~define.setup~ file, type the key sequence \verb~C-c C-v t~ in Emacs (with
the default keymap).  The first line makes sure that the file can be
edited in org-mode despite the file being named with an extension of
\verb~.setup~.

\begin{verbatim}
# -*- mode: org -*-
#+MACRO: when-fmt (eval (when (org-export-derived-backend-p org-export-current-backend '$1) "$2"))
#+MACRO: preamble {{{when-fmt(html,\\($1\\))}}}{{{when-fmt(latex,\n#+LATEX_HEADER: $1\n)}}}
#+MACRO: def {{{preamble(\\def$1{$2})}}}
#+MACRO: newcommand {{{preamble(\\newcommand{$1}$3{$2})}}}
#+MACRO: renewcommand {{{preamble(\\renewcommand{$1}$3{$2})}}}
#+MACRO: newenvironment {{{preamble(\\newenvironment{$1}$4{$2}{$3})}}}
#+MACRO: renewenvironment {{{preamble(\\renewenvironment{$1}$4{$2}{$3})}}}
\end{verbatim}

\section{Preliminary Macros}
\label{sec-1}

\subsection{when-fmt}
\label{sec-1-1}

This is inspired by the \verb~if-latex-else~ macro under the `Advanced'
heading here: \url{https://github.com/fniessen/org-macros}. Apparently,
Org mode will evalute Emacs Lisp code in macros, although I have not
yet found any documentation that explains \emph{why} this works.

The \verb~when~ form is like \verb~if~, except it only returns a string when the
condition is true, returning \verb~nil~ instead when it is false.  We use
\verb~when~ because we want to perform an action for \LaTeX{} and HTML
formats, and we do not want to assume that the user wants the same
behavior for all non-\LaTeX{} or non-HTML formats.

\begin{verbatim}
#+MACRO: when-fmt (eval (when (org-export-derived-backend-p org-export-current-backend '$1) "$2"))
\end{verbatim}

Since the parameter \verb~$2~ is inside quotes, it will be necessary to
double any (literal) backslashes, despite the fact that Org mode
macros do not normally require this (except between parameters).

\subsection{preamble}
\label{sec-1-2}

Using the \ref{sec-1-1} macro, we wrap HTML output in \verb~\(...\)~ so that \href{http://docs.mathjax.org/en/latest/tex.html#defining-tex-macros}{the
MathJax library will recognize that it should process them}.  So long
as we only use this to define \LaTeX{} macros, MathJax will not generate
any spurious output.  In \LaTeX{} output, we use the \verb~#+LATEX_HEADER:~
Org mode syntax to ensure that it is put in the proper \LaTeX{} preamble.

\begin{verbatim}
#+MACRO: preamble {{{when-fmt(html,\\($1\\))}}}{{{when-fmt(latex,\n#+LATEX_HEADER: $1\n)}}}
\end{verbatim}

Note: the \verb~\n~ before \verb~#+LATEX_HEADER:~ and after the parameter \verb~$1~
are an \emph{attempt} to ensure that the command starts and ends with a new
line, so that Org mode commands are recognized correctly.  This is not
bulletproof: if you put a newline \emph{inside} the command, that may still
foul things up.

\section{Definition Macros}
\label{sec-2}

Using the \ref{sec-1-2} macro, we specify macros that use the \TeX{} and \LaTeX{}
macros to define macros.  Once again, due to the use of an Emacs Lisp
string, it is necessary to double any literal backslashes provided to
these macros, as explained for the \ref{sec-1-1} macro.

\subsection{def}
\label{sec-2-1}

The plain \TeX{} command, which allows you to (re)define anything that
(MathJax's implementation of) \TeX{} can handle, including \LaTeX{}.  This
flexibility comes at the price of no warnings when you attempt to
define something new and it redefines something essential.  The first
argument is the control sequence to define, which may begin with a
backslash (escaped: \verb~\\~), but must not contain curly brackets
(\verb~{...}~); the second argument is what it should expand to,
automatically enclosed in curly brackets.

\begin{verbatim}
#+MACRO: def {{{preamble(\\def$1{$2})}}}
\end{verbatim}

\subsection{newcommand}
\label{sec-2-2}

The standard \LaTeX{} command for defining a macro, which \emph{must not}
already exist.  The first argument is the command name, which is
automatically enclosed in brackets.  The second argument is what it
should expand to, which is also automatically enclosed in brackets.
The third argument, if any, is inserted \emph{as is} between the first and
second argument; this lets you give a parameter count, or a parameter
count \emph{and} a default value for the first parameter, each in square
brackets (\verb~[...]~).

\begin{verbatim}
#+MACRO: newcommand {{{preamble(\\newcommand{$1}$3{$2})}}}
\end{verbatim}

\subsection{renewcommand}
\label{sec-2-3}

The \LaTeX{} command for redefining an existing macro, which \emph{must}
already exist.  The arguments are the same as for the \ref{sec-2-2}
macro.

\begin{verbatim}
#+MACRO: renewcommand {{{preamble(\\renewcommand{$1}$3{$2})}}}
\end{verbatim}

\subsection{newenvironment}
\label{sec-2-4}

The \LaTeX{} command for defining a new environment, which \emph{must not}
already exist.  The first, second, and third arguments are enclosed in
curly brackets in that order.  The fourth argument, if any, is
inserted (as is) between the first and second arguments, like the
third argument for the \ref{sec-2-2} macro.

\begin{verbatim}
#+MACRO: newenvironment {{{preamble(\\newenvironment{$1}$4{$2}{$3})}}}
\end{verbatim}

\subsection{renewenvironment}
\label{sec-2-5}

The \LaTeX{} command for redefining an existing environment, which \emph{must}
already exist.  The arguments are the same as for the \ref{sec-2-4}
macro.

\begin{verbatim}
#+MACRO: renewenvironment {{{preamble(\\renewenvironment{$1}$4{$2}{$3})}}}
\end{verbatim}

\section{Installation}
\label{sec-3}

To use this setup file, you only need to "tangle" this document from
within Emacs (press \verb~C-c C-v t~ in the default keymap), drop it into
the same directory as your Org mode document(s), and load it with
\begin{verbatim}
#+SETUPFILE: define.setup
\end{verbatim}
near the beginning of each file.

\section{Usage Examples}
\label{sec-4}

\subsection{{\bfseries\sffamily TODO} def}
\label{sec-4-1}

\subsection{newcommand (without parameters)}
\label{sec-4-2}


Say you are writing a book on Intermediate Algebra, and introducing
the vertex form of quadratic equations.  And you find yourself
referring to the equation \[\myvqeqn\] a lot, so you would like to
abbreviate it.  Easy!  Anywhere (almost) \emph{before} your first usage ---
we do it right before this text, to help keep it consistent with the
example text and explanation here --- just type
\begin{verbatim}
{{{newcommand(\\myvqeqn,y = 2(x-1)^2+1)}}}
\end{verbatim}
and use it like so:
\begin{verbatim}
The equation $\myvqeqn$ shows ...
\end{verbatim}
It will look like "The equation $\myvqeqn$ shows \ldots{}", in both HTML
and \LaTeX{}, the latter thanks to the use of \verb~#+LATEX_HEADER:~ in the
\ref{sec-1-2} macro.

\subsection{newcommand (with a parameter)}
\label{sec-4-3}


Say you are writing a book on Linear Algebra.  You will certainly be
writing a lot about the names of certain matrices, and you will
probably want to abbreviate the names of (at least) the most common
matrices you talk about.  For two reasons, you will want to use a
parameter for the matrix name:

\begin{enumerate}
\item you may be writing about several matrices, and it would be a waste
of effort to write a separate command for each one;
\item you want to ensure that your typographical treatment of the various
matrices is consistent, even if you change your mind later.
\end{enumerate}

So, you might define a command like this:
\begin{verbatim}
{{{newcommand(\\mat,\\mathbf{#1},[1])}}}
\end{verbatim}
and use it like this:
\begin{verbatim}
The matrices $\mat{A}$ and $\mat{B}$ represent ...
\end{verbatim}

Now, the result looks like "The matrices $\mat{A}$ and $\mat{B}$
represent \ldots{}", but if you want to change how it looks later, there is
only one definition that needs to be changed.

\subsection{{\bfseries\sffamily TODO} renewcommand}
\label{sec-4-4}

\subsection{{\bfseries\sffamily TODO} newenvironment}
\label{sec-4-5}

\subsection{{\bfseries\sffamily TODO} renewenvironment}
\label{sec-4-6}
% Emacs 24.5.1 (Org mode 8.2.10)
\end{document}