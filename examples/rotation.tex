% Created 2016-08-25 Thu 23:58
\documentclass[11pt]{article}
\usepackage[utf8]{inputenc}
\usepackage[T1]{fontenc}
\usepackage{fixltx2e}
\usepackage{graphicx}
\usepackage{longtable}
\usepackage{float}
\usepackage{wrapfig}
\usepackage{rotating}
\usepackage[normalem]{ulem}
\usepackage{amsmath}
\usepackage{textcomp}
\usepackage{marvosym}
\usepackage{wasysym}
\usepackage{amssymb}
\usepackage{hyperref}
\tolerance=1000
\newcommand{\mat}[1]{\mathbf{#1}}
\renewcommand{\vec}[1]{\mathbf{#1}}
\newcommand{\e}[1]{\mathbf{e}_{#1}}
\author{Brian C. Wells}
\date{\today}
\title{Derivation of the 2-D Rotation Matrix}
\hypersetup{
  pdfkeywords={},
  pdfsubject={},
  pdfcreator={Emacs 24.5.1 (Org mode 8.2.10)}}
\begin{document}

\maketitle
\tableofcontents

This document is an extended example for using \href{../define.org}{this literate program}.
It should be available in both HTML and PDF versions, as well as the
Org mode source code.

\section*{Introduction}
\label{sec-1}

There are many articles on the Internet (including the \href{https://en.wikipedia.org/wiki/Rotation_matrix}{rotation matrix
article on Wikipedia}) which \emph{state} that the transformation matrix for
a 2-dimensional rotation through an angle $\theta$ can be expressed as
\begin{equation*}
\begin{bmatrix}
\cos\theta & -\sin\theta \\
\sin\theta & \cos\theta
\end{bmatrix},
\end{equation*}
but few I am aware of show exactly how this matrix can be derived \emph{by
purely algebraic methods}.  It is easy to show that this follows from
the standard 2-D rotation transformation (we do that also here), but
that is not very helpful for a student who has not seen how that
transformation itself can be derived.  There are websites that try to
show why it is true visually, but that is not mathematically rigorous
and is again unhelpful for students who have difficulty with
visualization (or who have serious sight impairments).

After I wrote the original version of this article, I found \href{http://www.sunshine2k.de/articles/RotationDerivation.pdf}{another
article} that explains rotations in basically the same way I did.  I
have incorporated its use of $\alpha$ and $\beta$ for the angles;
originally, I used different variables.  However, it does not explain
how it goes from transformation to matrix, substitutes diagrams for
formal notation and descriptions, and shows more intermediate steps
than someone who has a good grasp of basic algebra and trigonometry
needs.  And it references Wikipedia for the trigonometric identities,
not even attempting to prove them.  So the present document is (for
better or worse) a little more formal, self-contained, and less
reliant on visual presentations.  It is also available in both HTML
and PDF formats, unlike the other article (PDF only).

Supposing only knowledge of high school algebra and trigonometry, and
some basic facts about vectors and matrices such as \href{https://en.wikipedia.org/wiki/Matrix_multiplication}{matrix
multiplication}, the 2-D rotation transformation (and its corresponding
matrix) can be derived algebraically.  A little knowledge of linear
algebra, particularly how to derive transformation matrices from
linear transformations, would also be helpful.  But I try to leave no
doubt that the derivation is correct (though a bit mysterious) even
for those who know nothing about that subject.

The derivation makes use of two trigonometric identities, which
readers are assumed to have seen (and subsequently forgotten).  In the
interest of completeness\footnote{Not to mention the fact that I have some difficulty remembering
the identities myself!}, a proof for these identities---making
use of \href{https://en.wikipedia.org/wiki/Euler\%2527s_formula}{Euler's formula}---is provided; but since this is not
particularly relevant to the main result, and some readers may be
sufficiently familiar with this derivation that they do not wish to
see it again, it has been deferred to an \hyperref[sec-3]{Appendix} (with another link
at the place where the identities are used).

\section*{Derivation}
\label{sec-2}

Suppose we have a vector $\vec{p} = \langle x, y \rangle = x\e{1} +
y\e{2}$, where $\e{1} = \langle 1, 0 \rangle$ and $\e{2} = \langle 0,
1 \rangle$ are the standard basis vectors.  The Cartesian $x$ and $y$
coordinates can be rewritten as \textbf{polar coordinates} thus:
\begin{align*}
x &= r \cos \alpha \\
y &= r \sin \alpha\tag{1}
\end{align*}
for some radius $r$ and angle $\alpha$ (not necessarily positive nor
unique), measured counterclockwise from the positive $x$-axis.

We want to rotate it through an angle $\beta$ (counterclockwise if
$\beta > 0$).  Since a rotation does not change the radius, but only
adds to the angle, the final vector will be $\vec{p'} = \langle x', y'
\rangle$ such that
\begin{align*}
x' &= r \cos(\alpha + \beta) \\
y' &= r \sin(\alpha + \beta)\tag{2}.
\end{align*}
We are looking for a $2\times{}2$ matrix $\mat{A}$ such that
\begin{equation*}
\vec{p'} = \mat{A}\vec{p}\tag{3},
\end{equation*}
when both $\vec{p}$ and $\vec{p'}$ are expressed in column form.
Using the identities
\begin{align*}
\cos(\alpha + \beta) &= \cos\alpha \cos\beta - \sin\alpha \sin\beta \\
\sin(\alpha + \beta) &= \sin\alpha \cos\beta + \cos\alpha \sin\beta,
\end{align*}
which we derive from Euler's formula in the \hyperref[sec-3]{Appendix} below, we combine
with the equations in $(2)$ to get
\begin{align*}
x' &= r \cos\alpha \cos\beta - r \sin\alpha \sin\beta \\
y' &= r \sin\alpha \cos\beta + r \cos\alpha \sin\beta.
\end{align*}
Substituting $x$ and $y$ for their equivalents from $(1)$, and
rearranging to put always $x$ before $y$, we find that the correct 2-D
rotation transformation is
\begin{align*}
x' &= x \cos\beta - y \sin\beta \\
y' &= x \sin\beta + y \cos\beta\tag{4}.
\end{align*}

To put this transformation into matrix form, we need apply it to the
standard basis vectors, then label these transformed standard basis
vectors as the columns of matrix $\mat{A}$.  For $\e{1}$ we get
\begin{align*}
x' &= 1 \cos\beta - 0 \sin\beta = \cos\beta \\
y' &= 1 \sin\beta + 0 \cos\beta = \sin\beta,
\end{align*}
and for $\e{2}$
\begin{align*}
x' &= 0 \cos\beta - 1 \sin\beta = -\sin\beta \\
y' &= 0 \sin\beta + 1 \cos\beta = \cos\beta.
\end{align*}
Labeling these results as columns of $\mat{A}$,
\begin{align*}
\mat{A} = \begin{bmatrix}
\cos\beta & -\sin\beta \\
\sin\beta & \cos\beta
\end{bmatrix}.
\end{align*}

To verify this as an appropriate way to construct $\mat{A}$, we
substitute this matrix and the general vector $\vec{p}$ (in column
form) into the right-hand side of $(3)$, then carry out the matrix
multiplication:
\begin{equation*}
\begin{bmatrix}
\cos\beta & -\sin\beta \\
\sin\beta & \cos\beta
\end{bmatrix}
\begin{bmatrix}
x \\
y
\end{bmatrix} =
\begin{bmatrix}
x \cos\beta - y \sin\beta \\
x \sin\beta + y \cos\beta
\end{bmatrix}.
\end{equation*}

This matches our previous results in $(4)$, so we know that this
matrix $\mat{A}$ is a correct representation of the 2-D rotation
transformation. QED

\section*{Appendix}
\label{sec-3}

Euler's formula tells us that $e^{i\theta} = \cos\theta +
i\sin\theta$, where $i = \sqrt{-1}$ is the unit of imaginary numbers.
Applying it to $\theta = \alpha + \beta$, we obtain
\begin{equation*}
e^{i(\alpha + \beta)} = \cos(\alpha + \beta) + i\sin(\alpha + \beta).
\end{equation*}
But since $e^{i(\alpha + \beta)} = e^{i\alpha} e^{i\beta}$, we also see that
\begin{align*}
e^{i(\alpha + \beta)} &= (\cos\alpha + i\sin\alpha)(\cos\beta + i\sin\beta) \\
                      &= (\cos\alpha \cos\beta - \sin\alpha \sin\beta) +
                         i(\sin\alpha \cos\beta + \cos\alpha \sin\beta).
\end{align*}
Equating the real and imaginary components of these two expressions
for $e^{i(\alpha + \beta)}$, we obtain the trigonometric identities
\begin{align*}
\cos(\alpha + \beta) &= \cos\alpha \cos\beta - \sin\alpha \sin\beta \\
\sin(\alpha + \beta) &= \sin\alpha \cos\beta + \cos\alpha \sin\beta,
\end{align*}
as desired.
% Emacs 24.5.1 (Org mode 8.2.10)
\end{document}